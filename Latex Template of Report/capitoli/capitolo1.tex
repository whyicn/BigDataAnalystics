\chapter{Business Understanding And Data Understanding}
\section{Business Understanding}

According to the dataset, predicting customer income can provide better benefits to the company. By clustering customers, it is possible to analyze their behavior and gain insights into their preferences and needs. This can help companies tailor their products and services to meet the specific needs of their customers. Such an approach can lead to increased customer satisfaction and loyalty, which can ultimately translate into higher profits for the company. 

\subsection{Data mining Goals}
%(Describe the intended outputs that enables the achievement of the project objectives) 
The goal is to create a tool capable of estimating new properties that will be put on the market, understanding the importance of the parameters that contribute to the economic evaluation of the property, in order to have a more accurate estimate of the value of the property based on the parameters that allow its value to be surveyed. For each property there is no right or wrong answer to its market value and it is not even possible to make a comparison with other properties present, albeit close to rent, as each property has different characteristics from each other.

\subsection{Data mining Success Criteria}
%(Define the criteria for a successful outcome) 
The success criteria should be based on the data mining goal determined earlier and should be used to formulate benchmarks for success. The methods for model assessment should be described, and benchmarks for evaluating success should be defined. Specific numbers should be provided, and subjective measurements should be determined as best as possible. 

The phases that we are going to perform by dividing the work by trying different methodologies in order to create a model as close as possible to the training set provided.
The distributing the hours in this way:
\begin{itemize}
\item Business Understanding: 5\% of total hours;
\item Data Understanding: 10\%;
\item Data Preparation: 60\%; 
\item Modeling: 20\%;
\item Evaluation: 5\%.
\end{itemize}

\subsection{Data Description Report}
% Describe the data that has been acquired including its format, its quantity

The dataset has the following characteristics: 
\begin{itemize}
    \item Multivariate;
    \item 2240 rows representing the number of records;
    \item 29 columns representing the number of attributes in the dataset;
\end{itemize}

The details of the dataset can be found at   

\href{kaggle}{https://www.kaggle.com/datasets/rodsaldanha/arketing-campaign}

